%%%%%%%%%%%%%%%%%%%%%%%%%%%%%%%%%%%%%%%%%%%%%%%%%%%%%%%%%%%%%%%%%%
%%%%%%%%%%%%%%%%%%%%%%%%%%%%%%%%%%%%%%%%%%%%%%%%%%%%%%%%%%%%%%%%%%
%Packages
\documentclass[10pt, a4paper]{article}
\usepackage[top=2cm, bottom=2cm, left=2cm, right=2cm]{geometry}
\usepackage{amsmath,amsthm,amsfonts,amssymb,amscd, fancyhdr, color, comment, graphicx, environ}
\usepackage{float}
\usepackage{mathrsfs}
\usepackage[math-style=ISO]{unicode-math}
\setmathfont{TeX Gyre Termes Math}
\usepackage{lastpage}
\usepackage[dvipsnames]{xcolor}
\usepackage[framemethod=TikZ]{mdframed}
\usepackage{enumerate}
\usepackage[shortlabels]{enumitem}
\usepackage{fancyhdr}
\usepackage{indentfirst}
\usepackage{listings}
\usepackage{sectsty}
\usepackage{thmtools}
\usepackage{shadethm}
\usepackage{hyperref}
\usepackage{setspace}
\hypersetup{
    colorlinks=true,
    linkcolor=blue,
    filecolor=magenta,      
    urlcolor=blue,
}
%%%%%%%%%%%%%%%%%%%%%%%%%%%%%%%%%%%%%%%%%%%%%%%%%%%%%%%%%%%%%%%%%%
%%%%%%%%%%%%%%%%%%%%%%%%%%%%%%%%%%%%%%%%%%%%%%%%%%%%%%%%%%%%%%%%%%
%Environment setup
\mdfsetup{skipabove=\topskip,skipbelow=\topskip}
\newrobustcmd\ExampleText{%
An \textit{inhomogeneous linear} differential equation has the form
\begin{align}
L[v ] = f,
\end{align}
where $L$ is a linear differential operator, $v$ is the dependent
variable, and $f$ is a given non−zero function of the independent
variables alone.
}

\mdtheorem[style=theoremstyle]{Problem}{Problem}
\newenvironment{Solution}{\textbf{Solution.}}

%%%%%%%%%%%%%%%%%%%%%%%%%%%%%%%%%%%%%%%%%%%%%%%%%%%%%%%%%%%%%%%%%%
%%%%%%%%%%%%%%%%%%%%%%%%%%%%%%%%%%%%%%%%%%%%%%%%%%%%%%%%%%%%%%%%%%
%Fill in the appropriate information below
\newcommand{\norm}[1]{\left\lVert#1\right\rVert}     
\newcommand\course{Course}                      % <-- course name   
\newcommand\hwnumber{1}                         % <-- homework number
\newcommand\Information{XXX/xxxxxxxx}           % <-- personal information
%%%%%%%%%%%%%%%%%%%%%%%%%%%%%%%%%%%%%%%%%%%%%%%%%%%%%%%%%%%%%%%%%%
%%%%%%%%%%%%%%%%%%%%%%%%%%%%%%%%%%%%%%%%%%%%%%%%%%%%%%%%%%%%%%%%%%
%Page setup
\pagestyle{fancy}
\headheight 35pt
\lhead{\today}
\rhead{\includegraphics[width=2.5cm]{IIITH_logo.png}} % <-- school logo(please upload the file first, then change the name here)
\lfoot{}
\pagenumbering{arabic}
\cfoot{\small\thepage}
\rfoot{}
\headsep 1.2em
\renewcommand{\baselinestretch}{1.25}       
\mdfdefinestyle{theoremstyle}{%
linecolor=black,linewidth=1pt,%
frametitlerule=true,%
frametitlebackgroundcolor=gray!20,
innertopmargin=\topskip,
}
%%%%%%%%%%%%%%%%%%%%%%%%%%%%%%%%%%%%%%%%%%%%%%%%%%%%%%%%%%%%%%%%%%
%%%%%%%%%%%%%%%%%%%%%%%%%%%%%%%%%%%%%%%%%%%%%%%%%%%%%%%%%%%%%%%%%%
%Add new commands here
\renewcommand{\labelenumi}{\alph{enumi})}
\newcommand{\Z}{\mathbb Z}
\newcommand{\R}{\mathbb R}
\newcommand{\Q}{\mathbb Q}
\newcommand{\NN}{\mathbb N}
\DeclareMathOperator{\Mod}{Mod} 
\renewcommand\lstlistingname{Algorithm}
\renewcommand\lstlistlistingname{Algorithms}
\def\lstlistingautorefname{Alg.}
%%%%%%%%%%%%%%%%%%%%%%%%%%%%%%%%%%%%%%%%%%%%%%%%%%%%%%%%%%%%%%%%%%
%%%%%%%%%%%%%%%%%%%%%%%%%%%%%%%%%%%%%%%%%%%%%%%%%%%%%%%%%%%%%%%%%%
%Begin now!



\begin{document}

\begin{titlepage}
    \begin{center}
        \vspace*{3cm}
            
        \Huge
        \textbf{Assignment - 1}
            
            
        \vspace{1.5cm}
        \Large
            
        \textbf{Siddik Ayyappa}                      % <-- author
        \\*{2020101089}
        \\*{CSE, UG2k20}
            
        \vfill
        
        \textbf{SMAI Homework Assignment}
            
        \vspace{1cm}
            
        \includegraphics[width=0.5\textwidth]{IIITH_logo.png}
        \\
        
        \Large
        
        \today
            
    \end{center}
\end{titlepage}

%%%%%%%%%%%%%%%%%%%%%%%%%%%%%%%%%%%%%%%%%%%%%%%%%%%%%%%%%%%%%%%%%%
%%%%%%%%%%%%%%%%%%%%%%%%%%%%%%%%%%%%%%%%%%%%%%%%%%%%%%%%%%%%%%%%%%
%Start the assignment now

%%%%%%%%%%%%%%%%%%%%%%%%%%%%%%%%%%%%%%%%%%%%%%%%%%%%%%%%%%%%%%%%%%
%New problem
\newpage
% Question - 1
\begin{Problem}
Give an example each of probability mass functions with finite and infinite ranges. Show that the conditions on PMF are satisfied by your example.
\end{Problem}
    
\begin{Solution}
Consider the example of the following PMF n: Poisson Random Variable\\
\begin{align*}
    p(n) &= \frac{\lambda^n}{n!}e^{-\lambda} &[\lambda > 0]\\
\end{align*}
This is an example of a probability mass function with an infinite range. As \\ n \rightarrow \infty \\p(n) \rightarrow 1 \\ Hence there are an infinitely many values for which p(n) $\neq$ 0 \\
\begin{align*}
    p(n) &\geq 0 &[\since \text{all the terms are positive}]\\
    &{}&{} [\text{condition (1) satisfied}]\\
    \sum_{n=1}^{\infty}p(n) &= \sum_{n=1}^{\infty}\frac{\frac{\lambda^n}{n!}}{e^{\lambda}}\\
    &= \frac{\sum_{n=1}^{\infty}\frac{\lambda^n}{n!}}{e^{\lambda}} &[\since \sum_{n=1}^{\infty}\frac{\lambda^n}{n!} = e^{\lambda}]\\
    &= \frac{e^{\lambda}}{e^{\lambda}} = 1\\
    \therefore \sum_{n=1}^{\infty}p(n) &= 1 &[\text{condition (2) satisfied}]
\end{align*}
\end{Solution} 


% Question - 2
\begin{Problem}
Show with complete steps that the variance of uniform density is given by equation 10. (Hint: use the expression for variance in equation 5.)
\end{Problem}
\begin{Solution}
We know that the variance of a density is given by $\sigma^2 = E((x-\mu)^2)$ where $\mu$ is the mean of the density.
\begin{align*}   
    \sigma^2 &= \int_{-\infty}^{\infty}((x-\mu^2)).p(x)dx\\
    &= \int_{-\infty}^{\infty}(x^2 - 2\mu x + \mu^2).p(x)dx\\
    &= \int_{-\infty}^{b}(x^2 - 2\mu x + \mu^2).p(x)dx + \int_{b}^{a}(x^2 - 2\mu x + \mu^2).p(x)dx + \int_{a}^{\infty}(x^2 - 2\mu x + \mu^2).p(x)dx\\
    &= 0 + \int_{a}^{b}(x^2 - 2\mu x + \mu^2).p(x)dx + 0 &[\because eq(9)]\\
    &= \int_{a}^{b}(x^2 - 2\mu x + \mu^2).p(x)dx\\
    &= \int_{a}^{b}(x^2).p(x)dx - 2\mu\int_{a}^{b}x.p(x)dx + \mu^2 &[\because eq(3)]\\
    &= \int_{a}^{b}(x^2).p(x)dx - 2\mu*\mu + \mu^2 &[\because eq(9)]\\
    &= \int_{a}^{b} x^2.p(x) - (\frac{a+b}{2})^2\\
    &= \int_{a}^{b} x^2.\frac{1}{b-a}.p(x)dx - \frac{a^2 + 2ab + b^2}{4}\\
    &= \bigg[ \frac{x^3}{3}*\frac{1}{b-a}\bigg]_{b}^{a} - \frac{a^2 + 2ab + b^2}{4}\\
    &= \bigg[ \frac{b^3 - a^3}{3}*\frac{1}{b-a}\bigg] - \frac{a^2 + 2ab + b^2}{4}\\
    &= \bigg[ \frac{(b-a)(b^2 + ab + a^2)}{3}*\frac{1}{b-a}\bigg]- \frac{a^2 + 2ab + b^2}{4}\\
    &= \bigg[ \frac{(b^2 + ab + a^2)}{3}\bigg]- \frac{a^2 + 2ab + b^2}{4}\\
    &= \bigg[ \frac{(4b^2 + 4ab + 4a^2)}{12}\bigg]- \frac{3a^2 + 6ab + 3b^2}{12}\\
    &= \frac{b^2 - 2ab + a^2}{12}\\
    \sigma^2 &= \frac{(b-a)^2}{12}\\
\end{align*}
Hence the variance of uniform density is $\sigma^2 = \frac{(b-a)^2}{12}$
\end{Solution}


% Question - 3
\begin{Problem}
Show examples of two density functions (draw the function plots) that have the same mean and variance, but clearly different distributions. Plot both functions in the same graph with different colours.
\end{Problem}

\begin{Solution}

\end{Solution}







% Question - 4
\begin{Problem}
Show that the alternate expression for variance given in equation 5 holds for discrete random variables as well.
\end{Problem}
\begin{Solution}
The equation for the variance of a discrete random variable is given by eq.4, which is
\begin{align*}
    Var(x) \equiv \sigma^2 &= E[(x-\mu^2)] = \sum_{i=1}^{n}(v_i - \mu)^2p(v_i)\\
    &= \sum_{i=1}^{n} (v_i^2 - 2\mu v_i + \mu^2)\\
    &= \sum_{i=1}^{n} v_i^2 p(v_i) - 2\mu\sum_{i=1}^{n}v_i p(v_i) + \mu^2\sum_{i=1}^{n}p(v_i) \\*&\because \sum_{i=1}^{n}v_i p(v_i) = \mu &\because\sum_{i=1}^{n}p(v_i) = 1\\
    &= \sum_{i=1}^{n} v_i^2 p(v_i) - 2\mu*\mu + \mu^2 &\because \sum_{i=1}^{n}f(v_i)p(v_i) = E(f(x))\\
    &= E[x^2] - \mu^2 &\because E[x] = \mu\\
    \sigma^2&= E[x^2] - (E[x])^2\\
\end{align*}
Hence the expression for variance given in equation 5 holds for discrete random variables as well.
\end{Solution}




%%%%%%%%%%%%%%%%%%%%%%%%%%%%%%%%%%%%%%%%%%%%%%%%%%%%%%%%%%%%%%%%%%
%Complete the assignment now
\end{document}

%%%%%%%%%%%%%%%%%%%%%%%%%%%%%%%%%%%%%%%%%%%%%%%%%%%%%%%%%%%%%%%%%%
%%%%%%%%%%%%%%%%%%%%%%%%%%%%%%%%%%%%%%%%%%%%%%%%%%%%%%%%%%%%%%%%%%
